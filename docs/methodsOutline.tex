%%% start preambling . . .  %%%
\documentclass{article}

% required 
\usepackage[hyphens]{url} % this wraps my URL versus letting it spill across the page, a bad habit LaTeX has

\usepackage{Sweave}
\usepackage{graphicx}
\usepackage{natbib}
\usepackage{amsmath}
\usepackage{textcomp}%amoung other things, it allows degrees C to be added
\usepackage{float}
\usepackage[utf8]{inputenc} % allow funny letters in citaions 
\usepackage[nottoc]{tocbibind} %should add Refences to the table of contents?
\usepackage{amsmath} % making nice equations 
\usepackage{listings} % add in stan code
\usepackage{xcolor}
\usepackage{capt-of}%alows me to set a caption for code in appendix 
\usepackage[export]{adjustbox} % adding a box around a map
\usepackage{lineno}
\linenumbers
% recommended! Uncomment the below line and change the path for your computer!
% \SweaveOpts{prefix.string=/Users/Lizzie/Documents/git/teaching/demoSweave/Fig.s/demoFig, eps=FALSE} 
%put your Fig.s in one place! Also, note that here 'Fig.s' is the folder and 'demoFig' is what each 
% Fig. produced will be titled plus its number or label (e.g., demoFig-nqpbetter.pdf')
% make your captioning look better
\usepackage[small]{caption}

\usepackage{xr-hyper} %refer to Fig.s in another document
\usepackage{hyperref}

\setlength{\captionmargin}{30pt}
\setlength{\abovecaptionskip}{0pt}
\setlength{\belowcaptionskip}{10pt}

% optional: muck with spacing
\topmargin -1.5cm        
\oddsidemargin 0.5cm   
\evensidemargin 0.5cm  % same as oddsidemargin but for left-hand pages
\textwidth 15.59cm
\textheight 21.94cm 
% \renewcommand{\baselinestretch}{1.5} % 1.5 lines between lines
\parindent 0pt		  % sets leading space for paragraphs
% optional: cute, fancy headers
\usepackage{fancyhdr}
\pagestyle{fancy}
\fancyhead[LO]{Draft early 2022}
\fancyhead[RO]{Temporal Ecology Lab}
% more optionals! %

%\graphicspath{ {./DoseResponse_WriteupImages/} }% tell latex where to find photos 
\externaldocument[supp-]{Traitors_Manuscript_supp}

%%% end preambling. %%%

\begin{document}
\input{methodsOutline-concordance}


\section*{Materials and Methods}
\subsection*{Field sampling}
\begin{enumerate}
\item General pop description
\begin{enumerate}
\item Combined \emph{in situ} trait data with budburst data from two growth chamber cutting experiments
\item Traits were measured across four eastern populations --- Harvard Forest, Massachusetts, USA (42.55$^{\circ}$N, 72.20$^{\circ}$W) and St. Hippolyte, Quebec, Canada (45.98$^{\circ}$N, 74.01$^{\circ}$W) and four western population --- Smithers (54.78$^{\circ}$N, 127.17$^{\circ}$W) and E.C. Manning Park (49.06$^{\circ}$N, 120.78$^{\circ}$W)
\item Growth chamber study --- cuttings collected from the most southern and northern populations in each transect
\item Both datasets span latitudinal gradient of 4-6$^{\circ}$
\end{enumerate}

\item Describing spp
\begin{enumerate}
\item Across populations --- measured 47 species --- 28 eastern transect and 22 western transect
\item Selected spp that were most abundant in forest communities
\item Eastern transect: 13 shrubs and 15 trees
\item Western transect: 18 shrubs and 4 trees
\item Three species occurred in both transect
\end{enumerate}
\end{enumerate}

\subsection*{Functional traits}

\begin{enumerate}
\item General sampling info
\begin{enumerate}
\item Traits were measured in the summer prior to each growth chamber study.
\item Eastern transect: traits measured summer 2015 
\item Western transect: traits measured May 29 to July 30, 2019
\item For each --- measured traits for X-10 healthy, adult, individuals of each species
\end{enumerate}

\item Trait sampling --- structural/wood traits
\begin{enumerate}
\item Measured a total of five traits: height (n = 1317), DBH (n = 1220), SSD (n = 1359), LMA (n = 1345), LNC (n = 1351)
\item Traits were measured according to recommendations of Perez-Harguindeguy2013.
\item Height was measured using a range finder (XX brand etc.).
\item Diameter at breast height was measured 1.42 m from the ground.
\item Wood volume was measured within 12 hours of collection --- 10 cm sample taken from tree branches or shrub primary stems --- measured using the water displacement method
\item Branch segments were dried at 105$^{\circ}$C for 24h and weighted.
\end{enumerate}

\item Trait sampling --- leaf traits
\begin{enumerate}
\item Leaf traits were measured for five haphazardly selected, fully expanded, and hardened leaves 
\item High resolution scans of leaf area taken using Canon flatbed scanner within 12 hours of collection
\item Leaf area was then quantified using the ImageJ software
\item Leaves were dried for 48h at 70$^{\circ}$C and weighed using a precision balance.
\end{enumerate}
\end{enumerate}

\subsection*{Growth chamber study}
\begin{enumerate}
\item General information
\begin{enumerate}
\item To collect branch cuttings for growth chamber study --- returned to our highest and lowest latitude populations in each transect
\item Eastern transect: Hippolyte and Harvard Forested visited again in 20-28 January for growth chamber study --- study conducted at Arnold Arboretum
\item Western transect: traits measured May 29 to July 30, 2019 --- Smithers and Manning park visited again in 19-28 October, 2019 --- study conducted at UBC
\end{enumerate}

\item Treatments and duration
\begin{enumerate}
\item For both growth chamber studies --- 8 distinct treatments:
\item two levels of chilling --- no additional chilling or 30 days at 4$^{\circ}$C = eastern study, and 30 days or 70 days of chilling at 4$^{\circ}$C for our western study --- all dark
\item two levels of forcing—a cool regime of 15:5$^{\circ}$C and a warm regime of 20:10$^{\circ}$C
\item two photoperiods of either 8 or 12 hours
\item For detailed disucssion of study differences see Loughnan et al. (phenology paper)
\end{enumerate}

\item Data collection --- BBCH
\begin{enumerate}
\item Phenological stages were based on BBCH scale adapted for our specific species
\item Observations made every 1-3 days for each sample
\item stages were recorded up to full leaf expansion --- limited our analysis to budburst 
\item Eastern study lasted for 82 days, 19320 observations---add Sexpr obj
\item Western study lasted for 113 days, 47844 observations
\end{enumerate}

\end{enumerate}

\subsection*{Statistical Anlaysis}

\begin{enumerate}
\item Introduce approach and why it is useful:
\begin{enumerate}
\item Our analysis combined the trait data with the budburst data from the growth chamber study
\item Built joint model for each trait---directly models the relationship between leaf and structural traits and budburst 
\item Approach carries through uncertainty between trait and phenology data---combines observational trait data with experimental phenology data
\end{enumerate}

\item Begin to describe the model:
\begin{enumerate}
\item hierarchical linear model---partitioning variation of individual observations (\emph{i}) of a given trait value ($y_{\text{trait}[i]}$) to the effects of species (\emph{sp id}), and population-level differences arising from transects (\emph{transect id}) or the interaction between transects and latitude (\emph{transect} $\times$ \emph{latitude}), and residual variation ($\sigma_{\text{trait}}$), sometimes called `measurement error').
\item Transect included as a dummy variable and latitude as a continuous variable 
\item Latitude values z-scored 
\item Most traits were modeled using their natural units---except LMA was rescaled by 100
\end{enumerate}

\item describe trait part of model
\begin{enumerate}
\item Model give unique estimates for each species ($\alpha_{\text{sp[sp id]}}$) and estimate of variance---partial pooling---controls for variation in the number of trait estimates per spp and trait variability 
\item Estimate for differences between transect ($\beta_{transect}$) and the interaction between transects and latitude ($\beta_{transect \times latitude}$) 
\item These species-level estimates of traits become predictors of species-level estimates of each cue--- ($\beta_{\text{force[sp]}}$, $\beta_{\text{chill[sp]}}$, $\beta_{\text{photo[sp]}}$) 
\end{enumerate}

Trait model:
\begin{align}
\mu_{trait} & = \alpha_{\text{grand trait}} + \alpha_{\text{sp[sp id]}} + \beta_{\text{transect}} +\beta_{\text{transect} \times \text{latitude}} \\
\beta_{transect} &  \sim normal( \mu_{transect}, \sigma_{\text{transect}}) \nonumber \\
\beta_{\text{transect} \times \text{latitude}} &  \sim normal( \mu_{\text{transect} \times \text{latitude}}, \sigma_{\text{transect} \times \text{latitude}}) \nonumber 
\end{align}

Cue part:
\begin{align}
\beta_{\text{chill[sp]}} & = \alpha_{\text{chill}[sp]} + \beta_{\text{trait}.\text{chill}} \times \alpha_{\text{trait sp[sp]}} \nonumber \\
\beta_{\text{force[sp]}} & = \alpha_{\text{force}[sp]} + \beta_{\text{trait}.\text{force}} \times \alpha_{\text{trait sp[sp]}} \\
\beta_{\text{photo[sp]}} & = \alpha_{\text{photo}[sp]} + \beta_{\text{trait}.\text{photo}} \times \alpha_{\text{trait sp[sp]}} \nonumber 
\end{align}

\item Explain phenology part of model
\begin{enumerate}
 \item Across each spp---get esti of overall effect of each trait on cue ($\beta_{\text{trait}.\text{chill}}$, $\beta_{\text{trait}.\text{force}}$, $\beta_{\text{trait}.\text{photo}}$)
\item and estimate of species-level cue variation not explained by traits ($\alpha_{\text{chill}[sp]} $, $\alpha_{\text{force}[sp]}$, $\alpha_{\text{photo}[sp]}$) = estimate of how well trait effects explain species-level differences
\item Also get estimates of species responses to cues ($C_i$, $F_i$, $P_i$, respectively---z-scored)---with residual variation across species ($\alpha_{\text{pheno[sp]}}$) and observations ($\sigma_{\text{pheno}}$)
\item Partial pooling for residual variation across species and variation in cues not attributed to the trait
\end{enumerate}

\begin{align}
\mu_{pheno} & = \alpha_{\text{pheno[sp]}}+ \beta_{\text{chill[sp]}} \times C_i + \beta_{\text{force[sp]}}\times F_i + \beta_{\text{photo[sp]}} \times P_i\\
y_{\text{pheno[i]}} & \sim normal( \mu_{pheno}, \sigma_{\text{pheno}}) \nonumber 
\end{align}

\begin{align}
\alpha_{\text{pheno}} \sim normal(\mu_{\alpha_{\text{pheno}}},\sigma_{\alpha_{\text{pheno}}}) \\
\alpha_{\text{force}} \sim normal(\mu_{\alpha_{\text{force}}},\sigma_{\alpha_{\text{force}}}) \nonumber \\
\alpha_{\text{chill}} \sim normal(\mu_{\alpha_{\text{chill}}},\sigma_{\alpha_{\text{chill}}}) \nonumber \\
\alpha_{\text{photo}} \sim normal(\mu_{\alpha_{\text{photo}}},\sigma_{\alpha_{\text{photo}}}) \nonumber
\end{align}

\item General info about model checks etc.:
\begin{enumerate}
\item weakly informative priors
\item validated model priors using prior predictive checks
\item model was coded in the Stan programming language and fit using the rstan package cite rstan2018
\item Four chains (4000-6000 total sampling iterations), and all models met basic diagnostic checks, including no divergences, high effective sample size (\emph{$n_{eff}$}), and $\hat{R}$ close to 1---LMA model still has low ESS---issue with neff---otherwise this is true
\item 90\% UI given in text

\end{enumerate} 

\end{enumerate} 





\end{document}

