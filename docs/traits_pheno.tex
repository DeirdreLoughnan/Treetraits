\documentclass[11pt,a4paper,oneside]{article}
\renewcommand{\baselinestretch}{1.8}
% \renewcommand*{\thefootnote}{\fnsymbol{footnote}}
\usepackage{sectsty,setspace} 
\usepackage[top=1.00in, bottom=1.0in, left=1in, right=1in]{geometry} 
\usepackage{graphicx}
\usepackage{epstopdf}
\usepackage{amsmath,latexsym,amssymb,wasysym}
\usepackage{natbib}
\usepackage{lineno}
\usepackage{todonotes}
\usepackage{hyperref}

%Potential titles:
\title{Traits and phenology: understanding the functional response of temperate woody plant species to climate} % traits chapter
%\title{Climate or traits: understanding the drivers of spring phenology in temperate woody species}

\author{D. Loughnan}
 
\begin{document}
\bibliographystyle{refs/bibstyles/Science.bst}% 

\maketitle
 
\section*{Summary}
\par Notable shifts in the timing of spring phenological events, such as budburst and flowering, have been observed for many temperate plant species. These changes are thought to reflect concurrent changes in climate, such as increasing temperatures and earlier snowmelt dates \cite{ Anderson2012}. Across a community, these changes in phenology can alter trophic interactions and carbon sequestration, altering the local species assemblage and ecosystem services \cite{Kharouba2018, Cleland:2007}. Individual species, however, vary in the magnitude and direction of their phenological changes \cite{Fitter2002,Dunnell2011, Konig2018}, suggesting that other factors in addition to climate may be contributing to species-specific performance.

\par In recent years functional ecology has made considerable strides towards understanding associates between vegetative and reproductive traits and plant performance \cite{McGill2006}.  The role of phenology in shaping these responses has gone largely unaddressed, despite classic studies illustrating its relationship to functional traits, such as wood structure \cite{Lechowich1984}. More recent studies in grass species have also found further evidence for a relationship between environmental conditions and functional traits associated with different strategies of resource use and competitive abilities \cite{Konig2018; Sun2011}. This currently limited body of work suggests that early leafing species should exhibit traits associated with faster growth, but lower competitive abilities, and reduced investment in plant tissue. This strategy enables these species to recover from extreme climate events, such as late spring frost events, at a lower cost. Later flowering species, however, are hypothesized to possess traits associated with greater competitive abilities, such as greater heights and stem or wood density in order to compete for light. 

\par In temperate woody plant species, spring phenological events relate to three primary climate conditions: winter chilling, spring forcing, and photoperiod \cite{Chuine2016}. The relative importance of each of these factors defines when a species will initiate activity in the spring and the abiotic and biotic community it experiences. The relative importance and the combined effects of these three climate conditions has been shown to vary both regionally and across species (CITATION). Whether this variation is reflected in suites of functional traits associated with commonly observed growth strategies has yet to be explored for temperate woody species. In this study, we predict that species that budburst earlier in response to low chilling and high forcing temperatures, and short photoperiods, will also characterized as having leaves with lower leaf mass areas, higher carbon to nitrogen ratios, and lower wood densities. Species that exhibit a more conservative growth strategy should require greater chilling and photoperiod requirements, while exhibiting traits associated with greater competitive ability and investment in tissue, such as greater height, wood density, lower leaf mass areas, and smaller carbon to nitrogen ratios.

By drawing these associations between functional traits and phenological responses under variable climate conditions, we can further develop a mechanistic understanding of how budburst of temperate woody plant species varies with climate, and begin to predict how species temporal niche is likely to change with climate change. 

\section*{Methods}
\par To identify functional traits that covary with phenology, traits of dominant woody species were measured across an eastern transect consisting of four sites from Quebec, Canada to Massachusetts, USA and a western transect of four sites in British Columbia, Canada, both of which span a latitudinal gradient of approximately 5 degrees. Each site consisted of multiple forest stands, the size of which depended on the area needed to meet our desired sample size.  Trait data was collected during the 2015 growing season for the eastern transect, sampling up to ten healthy, adult individuals from a pool of 28 temperate woody species. The western transect was sampled during the 2019 growing season, with trait data collected for 26 species. 

\par For each individual we measured height and stem diameter in the field and collected leaf and wood tissue to quantify leaf mass area (LMA), the ratio of carbon and nitrogen (C:N), and branchwood specific density. Individuals for each of our focal species were haphazardly selected, depending on their abundance and accessibility in the field. Height was defined as the distance from the ground to the top of the main photosynthetic tissue, or tree crown. The distance from the observer to the tree and the treetop was measured using a laser range finder and used to calculate height using Pythagorean theorem. At the same time, stem diameter was measured either at breast height (1.37 m) for trees, or using digital callipers, we measured the stem base for woody shrubs not tall enough to measure at breast height.  We also removed a portion of a terminal branch from each individual and immediately placed both leaf and wood samples in sealed plastic bags in a cooler. On the same day as collection, five fully expanded and hardened leaves were selected and scanned in colour, at 300-600dpi. To preserved leaves during transport back to the lab, we stored them in plant presses and immediately upon returning to the lab placed the leaves in a drying oven. All leaves were dried for 72h at 70C. The software ImageJ was used to calculate the area of the leaves and LMA calculated as the ratio of the leaf mass over its area when fresh \cite{Perez-Harguindeguy2013}. Finally, to quantify branchwood density, we collected a 10cm long segment stem from the same branch used for the leaf collection. A consistent and representative anterior section of the terminal end was cut for each individual, thereby allowing the natural variation in stem diameter across species to be accommodated. Within 24 hours of sample collection we quantified the volume of each stem using the water displacement method \cite{Perez-Harguindeguy2013}. Upon returning to the lab, stems were dried at 101C for 72h and weighted. Wood density was calculated as the dry mass of the stem over its fresh volume.  


\section*{In this chapter, I will test the following questions:}
\begin{enumerate}
\item Do woody plants species express the predicted suites of traits that vary consistently with phenology? 
\item How do these suites of traits vary across latitude, particularly in response to differences in photoperiod and winter chilling?
\item Are there differences between these trends in eastern and western temperate forests, or are the observed trends consistent for species of the same genus and dependent on phylogenetic distance?

\end{enumerate}
\par %Thes

\section*{Milestones}

\textbf{September to mid October}
\par Finish trait quantification:
\begin{itemize}
\item ImageJ
\item Weighing wood tissue
\item Grind leaves
\item Encapsulate ground tissue
\item Get data on xylem structure 
\end{itemize}
\par Build better test data

\textbf{Mid October to March}
\par Growth chamber study for phenology data
\par Learn about joint models

\textbf{December}
\par Have good testdata

\textbf{January}
\par Test model using eastern dataset

\textbf{April}
\par Test model using total dataset

\textbf{May}
\par Have results and figure

\bibliography{refs/Temp.bib}


\end{document}  