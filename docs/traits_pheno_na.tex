% This code is taken from OSPREE/photoperiod 
%%%%%%%%%%%%%%%%%%%%%%%%%%%%%%%%%%%%%%START PREAMBLE THAT IS THE SAME FOR ALL EXAMPLES
\documentclass{article}\usepackage[]{graphicx}\usepackage[]{color}
%% maxwidth is the original width if it is less than linewidth
%% otherwise use linewidth (to make sure the graphics do not exceed the margin)
\makeatletter
\def\maxwidth{ %
  \ifdim\Gin@nat@width>\linewidth
    \linewidth
  \else
    \Gin@nat@width
  \fi
}
\makeatother

\definecolor{fgcolor}{rgb}{0.345, 0.345, 0.345}
\newcommand{\hlnum}[1]{\textcolor[rgb]{0.686,0.059,0.569}{#1}}%
\newcommand{\hlstr}[1]{\textcolor[rgb]{0.192,0.494,0.8}{#1}}%
\newcommand{\hlcom}[1]{\textcolor[rgb]{0.678,0.584,0.686}{\textit{#1}}}%
\newcommand{\hlopt}[1]{\textcolor[rgb]{0,0,0}{#1}}%
\newcommand{\hlstd}[1]{\textcolor[rgb]{0.345,0.345,0.345}{#1}}%
\newcommand{\hlkwa}[1]{\textcolor[rgb]{0.161,0.373,0.58}{\textbf{#1}}}%
\newcommand{\hlkwb}[1]{\textcolor[rgb]{0.69,0.353,0.396}{#1}}%
\newcommand{\hlkwc}[1]{\textcolor[rgb]{0.333,0.667,0.333}{#1}}%
\newcommand{\hlkwd}[1]{\textcolor[rgb]{0.737,0.353,0.396}{\textbf{#1}}}%
\let\hlipl\hlkwb

\usepackage{framed}
\makeatletter
\newenvironment{kframe}{%
 \def\at@end@of@kframe{}%
 \ifinner\ifhmode%
  \def\at@end@of@kframe{\end{minipage}}%
  \begin{minipage}{\columnwidth}%
 \fi\fi%
 \def\FrameCommand##1{\hskip\@totalleftmargin \hskip-\fboxsep
 \colorbox{shadecolor}{##1}\hskip-\fboxsep
     % There is no \\@totalrightmargin, so:
     \hskip-\linewidth \hskip-\@totalleftmargin \hskip\columnwidth}%
 \MakeFramed {\advance\hsize-\width
   \@totalleftmargin\z@ \linewidth\hsize
   \@setminipage}}%
 {\par\unskip\endMakeFramed%
 \at@end@of@kframe}
\makeatother

\definecolor{shadecolor}{rgb}{.97, .97, .97}
\definecolor{messagecolor}{rgb}{0, 0, 0}
\definecolor{warningcolor}{rgb}{1, 0, 1}
\definecolor{errorcolor}{rgb}{1, 0, 0}
\newenvironment{knitrout}{}{} % an empty environment to be redefined in TeX

\usepackage{alltt}

%Required: You must have these
\usepackage{Sweave}
\usepackage{graphicx}
\usepackage{tabularx}
\usepackage{hyperref}
\usepackage{natbib}
\usepackage{pdflscape}
\usepackage{array}
\usepackage{gensymb}
%\usepackage[backend=bibtex]{biblatex}
%Strongly recommended
  %put your figures in one place
%\SweaveOpts{prefix.string=figures/, eps=FALSE} 
%you'll want these for pretty captioning
\usepackage[small]{caption}

\setkeys{Gin}{width=0.8\textwidth}  %make the figs 50 perc textwidth
\setlength{\captionmargin}{30pt}
\setlength{\abovecaptionskip}{10pt}
\setlength{\belowcaptionskip}{10pt}
% manual for caption  http://www.dd.chalmers.se/latex/Docs/PDF/caption.pdf

%Optional: I like to muck with my margins and spacing in ways that LaTeX frowns on
%Here's how to do that
 \topmargin -1.5cm        
 \oddsidemargin -0.04cm   
 \evensidemargin -0.04cm  % same as oddsidemargin but for left-hand pages
 \textwidth 16.59cm
 \textheight 21.94cm 
 %\pagestyle{empty}       % Uncomment if don't want page numbers
 \parskip 7.2pt           % sets spacing between paragraphs
 %\renewcommand{\baselinestretch}{1.5} 	% Uncomment for 1.5 spacing between lines
\parindent 0pt% sets leading space for paragraphs
\usepackage{setspace}
%\doublespacing

%Optional: I like fancy headers
%\usepackage{fancyhdr}
%\pagestyle{fancy}
%\fancyhead[LO]{How do climate change experiments actually change climate}
%\fancyhead[RO]{2016}
 
%%%%%%%%%%%%%%%%%%%%%%%%%%%%%%%%%%%%%%END PREAMBLE THAT IS THE SAME FOR ALL EXAMPLES

%Start of the document
\IfFileExists{upquote.sty}{\usepackage{upquote}}{}
\begin{document}
%\SweaveOpts{concordance=FALSE}
%\SweaveOpts{concordance=TRUE}

\title{Traits and phenology: understanding the functional response of temperate woody plant species to cliamte change} % traits chapter

\author{D. Loughnan}
%\date{\today}
\maketitle  %put the fancy title on
%\tableofcontents      %add a table of contents
%\clearpage
%goal is NCC Perspective
%Need to submit "brief synopsis through our online submission system before preparing a manuscript for formal submission. The synopsis should outline the topics that will be covered, list any recent, key publications in the area, and state the last time the topic was reviewed (if it has been reviewed previously)."
%should be presented using simple prose, avoiding excessive jargon and technical detail.
%3,000–5,000 words and typically include 4–6 display items (figures, tables or boxes). (we have 6)
%up to 100 references; citations should be selective. (we have 83 now)
%%%%%%%%%%%%%%%%%%%%%%%%%%%%%%%%%%%%%%%%%%%%%%%%%%%

%%%%%%%%%%%%%%%%%%%%%%%%%%%%%%%%%%%%%%%%%%%%%%%%%%%

\section*{Summary}

\par In recent decades, notable shifts in the timing of spring phenological events, such as budburst and flowering, have been observed for many temperate species. These changes are thought to reflect concurrent changes in climate, such as increasing temperatures and earlier snowmelt dates \cite{ Anderson2012}. Changes in phenology can alter trophic interactions and carbon sequestration, and may impact the species assemblage and structure of a community and its ecosystem services \cite{Kharouba2018, Cleland:2007}. Individual species, however, vary in the magnitude and direction of their phenological changes \cite{Fitter2002,Dunnell2011, Konig2018}, suggesting that other factors in addition to climate may be contributing to species-specific performance.

\par In recent years functional ecology has made considerable strides towards understanding associates between vegetative and reproductive traits and plant performance \cite{McGill2006}.  The role of phenology in shaping these responses has gone largely unaddressed, despite classic studies illustrating its relationship to functional traits \cite{Lechowich1984}. Phenology has been shown to be related to environmental conditions, with functional traits of grass species being associated with traits linked to resource use and competitive abilities \cite{Konig2018; SunFrelich2011}. This evidence suggests that early leafing species should exhibit traits associated with faster growth, but lower competitive abilities, and investment in plant tissue. This strategy enables these species to recover from extreme climate events, such as spring frost, at a lower cost. Later flowering species, however, are hypothesized to possess traits associated with greater competitive abilities, such as growing taller and having greater wood density in order to compete for light. 

\par In temperate woody plant species, spring phenological events relate to three primary climate conditions: winter chilling, spring forcing, and photoperiod \cite{Chuine2016}. The realtive importance of each of these factors defines when a species will initiate activity in the spring and the abiotic and biotic community it experiences. Species active early in the spring are at a greater risk of being damaged by frost, but may incur this cost if they are able to replace lost tissue more quickly. If this is the case, we predict these species to respond to high forcing temperatures, while requiring less chilling and be less photoperiod sensitive, producing leaves with lower leaf mass areas, higher C:N, and lower wood densities. Species that exhibit a more conservative growth strategy should have a greater chilling and photoperiod requirement, while exhibiting traits associated with greater competitive ability and investment in tissue, such as greater height, wood density, lower LMA, and smaller C:N.

By drawing these associations between functional traits and phenological responses under variable climate conditions, we can further develop a mechanistic understanding of how species phenology, and temporal niche, is likely to change with climate change. 


\section*{Methods}
\par To identify functional traits that covary with phenology, traits of dominant woody species were measured across four sites in British Columbia and from Quebec to Massachusett, spanning latitudinal gradients of approximatley 5 degrees. Each site could consist of multiple forest stands, depending on the area needed to meet our sample size per species.  At each site, we sampled up to ten healthy, adult individuals from a pool of 26 temperate woody species on the west coast and 28 specis on the east coast. For each individual we measured height and stem diameter in the field and collected leaf and wood tissue to quantify leaf mass area (LMA), the ratio of carbon and nitrogen (C:N), and branchwood specific density. 

\par Individuals for each of our focal species were haphazardly selected, depending on their abundance and accessibility in the field. Height was defined as the distance from the ground to the top of the main photosynthetic tissue, or tree crown. The distance from the observer to the tree and the treetop was measured using a laser range finder and used to calculate height using Pythagorean theorem. At the same time, stem diameter was measured either at breast height (1.37 m) for trees, or using digital calipers, we measured the stem base for woody shrubs not tall enough to measure at breast height.  We also removed a portion of a terminal branch from each individual and immediately placed in sealed plastic bags in a cooler. On the same day as collection, five fully expanded and hardened leaves were later selected and scanned in colour, at 300-600dpi. To preserved leaves during transport back to the lab, we stored them in plant presses and immediately upon returning to the lab placed the leaves in a drying oven. All leaves were dried for 72h at 70°C. The software ImageJ was used to calculate the area of the leaves and LMA calculated as the ratio of the leaf mass over its area when fresh \cite{Perez-Harguindeguy2013}. Finally, to quantify branchwood density, we collected a 10cm long segment stem from the same branch used for the leaf collection. A consistent and representative anterior section of the terminal was cut for each individual, thereby allowing the natural variation in stem diameter across species to be accommodated. Within 24 hours of sample collection we quantified the volume of each stem using the water displacement method \cite{Perez-Harguindeguy2013}. Upon returning to the lab, stems were dried at 101°C for 72h and weighted. Wood density was calculated as the dry mass of the stem over its fresh volume.  



\section*{In this chapter, I will test the following questions:}
\begin{enumerate}
\item Do woody plants species express specific suites of traits that vary consistently with phenology? 
\item How do these suites of traits vary across latitude, particularly in response to differences in photoperiod and winter chilling?
\item Are there differences between these trends in eastern and western temperate forests, or are the observed trends consistent for species of the same genus and dependent on phylogenetic distance?

\end{enumerate}
\par %Thes


\section*{Milestones}

\textbf{September to mid October}
\par Finish trait quantification:
\begin{itemize}
\item ImageJ
\item Weighing wood tissue
\item Grind leaves
\item Encapsulate ground tissue
\item Get data on xylem structure 
\end{itemize}
\par Build better test data

\textbf{Mid October to March}
\par Growth chamber study for phenology data
\par Learn about joint models

\textbf{December}
\par Have good testdata

\textbf{January}
\par Test model using eastern dataset

\textbf{April}
\par Test model using total dataset

\testbf{May}
\par Have results & figures for manuscript

\bibliographystyle{plain}% 
\bibliography{refs/outline_ref.bib}

%%%%%%%%%%%%%%%%%%%%%%%%%%%%%%%%%%%%%%%%
\end{document}
%%%%%%%%%%%%%%%%%%%%%%%%%%%%%%%%%%%%%%%%
